%%%%%%%%%%%%%%%%%%%%%%%%%%%%%%%%%%%%%%%%%%%%%%%%%%%%%%%%%%%%%%%%%%%%%%%%%%%%%%%% 
%%%%%%%%%%%%%%%%%%%%%%%%%%%%%%%%%%% PREAMBLE %%%%%%%%%%%%%%%%%%%%%%%%%%%%%%%%%%%
%%%%%%%%%%%%%%%%%%%%%%%%%%%%%%%%%%%%%%%%%%%%%%%%%%%%%%%%%%%%%%%%%%%%%%%%%%%%%%%% 


%%%%%%%%%%%%%%%%%%%%%%%%%%%%%%%%%%%%%%%%% 
%%%%%%%%%%%%%%%%% CLASS %%%%%%%%%%%%%%%%%
%%%%%%%%%%%%%%%%%%%%%%%%%%%%%%%%%%%%%%%%% 

\documentclass[11pt,letterpaper]{article}
\usepackage{geometry}
\usepackage{tabularx}
\usepackage{calendar}

% == margins
\addtolength{\hoffset}{-0.5in} \addtolength{\voffset}{-0.5in}
\addtolength{\textwidth}{1in} \addtolength{\textheight}{1.5in}

\usepackage{rotating}
\usepackage{lscape}

%%%%%%%%%%%%%%%%%%%%%%%%%%%%%%%%%%%%%%%% 
%%%%%%%%%%%%%%% PACKAGES %%%%%%%%%%%%%%%
%%%%%%%%%%%%%%%%%%%%%%%%%%%%%%%%%%%%%%%% 

\usepackage[T1]{fontenc}
\usepackage{hyperref}


%%%%%%%%%%%%%%%%%%%%%%%%%%%%%%%%%%%%%%%% 
%%%%%%%%%%%%%%% SETTINGS %%%%%%%%%%%%%%%
%%%%%%%%%%%%%%%%%%%%%%%%%%%%%%%%%%%%%%%% 

\title{\bf POL 245: Visualizing Data 
}

\author{
  \bf \Large Summer 2015 
}


\date{
  James Lo, Will Lowe (Instructors)
  \\
  Winston Chou, Elisha Cohen (Preceptors)
  \\
  Alex Tarr (QuantLab Coordinator)
  \\
  Kosuke Imai (Course Head) 
  \\ \vspace{.25in}
  Department of Politics, Princeton University
}

\newcommand\R{\textsf{\textbf{R}}}
\newcommand\Rst{\textsf{\textbf{RStudio}}}
\newcommand{\sitem}[1]{\item[\textit{#1}]}


%%%%%%%%%%%%%%%%%%%%%%%%%%%%%%%%%%%%%%%%%%%%%%%%%%%%%%%%%%%%%%%%%%%%%%%%%%%%%%% 
%%%%%%%%%%%%%%%%%%%%%%%%%%%%%%%%%%% CONTENT %%%%%%%%%%%%%%%%%%%%%%%%%%%%%%%%%%%
%%%%%%%%%%%%%%%%%%%%%%%%%%%%%%%%%%%%%%%%%%%%%%%%%%%%%%%%%%%%%%%%%%%%%%%%%%%%%%% 

\begin{document}

\maketitle

In this course, we consider ways to illustrate compelling stories hidden in a
blizzard of data.  Equal parts art, programming, and statistical reasoning, data
visualization is a critical tool for anyone doing analysis. In recent years,
data analysis skills have become essential for those pursuing careers in policy
advocacy and evaluation, business consulting and management, or academic
research in the fields of education, health, medicine, and social science. This
course introduces students to the powerful \R\ programming language and the
basics of creating data-analytic graphics in \R. From there, we use real
datasets to explore topics ranging from network data (like social interactions
on Facebook or trade between counties) to geographical data (like county-level
election returns in the US or the spatial distribution of insurgent attacks in
Afghanistan).  No prior background in statistics or programming is required or
expected.

\section*{Contact Information}
\begin{center}
  \begin{tabular}{r|  l l l l}
    Name         & James Lo      & William Lowe       & Kosuke Imai \\
    Office       & Corwin 029            & TBA       & Corwin 036     \\
    Email        &
    \href{mailto:jameslo@princeton.edu}{\texttt{jameslo@princeton.edu}}
    &
    \href{mailto:will.lowe@uni-mannheim.de}{\texttt{will.lowe@uni-mannheim.de}}
    &
    \href{mailto:kimai@princeton.edu}{\texttt{kimai@princeton.edu}}
%    \\
%    Office Hours & Fri: 1:30pm -- 3:00pm & TBA &
%    by appointment           
    \\
    \multicolumn{1}{c}{}\\
    Name         & Winston Chou      & Elisha Cohen       & Alex Tarr \\
    Office       & Corwin 127            & Corwin 127       & TBA     \\
    Email        &
    \href{mailto:wchou@princeton.edu}{\texttt{wchou@princeton.edu}}
    &
    \href{mailto:eacohen1@princeton.edu}{\texttt{eacohen1@princeton.edu}}
    &
    \href{mailto:atarr@princeton.edu}{\texttt{atarr@princeton.edu}}
%    \\
%    Office Hours & Fri: 1:30pm -- 3:00pm & Fri: 1:30pm -- 3:00pm &
%    by appointment           \\
 \end{tabular}
\end{center}

During our office hours we may be in either the office that is listed above or
in Corwin 023 (just across the hall) which has space for more students. If these
office hours do not fit your schedule, you should feel free to contact us
directly via email.  Also, do not forget about Piazza and QuantLab (see below)
where you can ask questions and receive answers back immediately.

\section*{Logistics}

\textit{The schedule during the first week deviates from this, details
  are below in the Course Outline section.}

\paragraph{Lectures.} Monday and Wednesday, 1:30pm--2:30pm, Sherrerd
Hall 101. To make lectures interactive, lecture slides will be posted
on Blackboard immediately {\it after} the lecture.  However, students
are expected to take notes during the lecture.

\paragraph{Precepts.} Tuesday and Thursday, 1:30pm--2:50pm, Frist
Campus Center 307, 309, and 329.  We ask you to bring your personal
laptop to precepts.

\paragraph{QuantLabs.} Monday, Tuesday, and Thursday, 3:00pm - 4:30pm,
Frist Campus Center 307, 309, and 329.  (following immediately after
the lecture on Mondays and the precepts on Tuesdays and Thursdays) in
the same room as your precepts.  You will be working with tutors on
review questions, practice exercises, and problem sets.  Bring your
own laptop to the QuantLabs.

\paragraph{Guest Lectures.} Friday, 10:30am--11:50am, Wallace Hall
300. These sessions occur during the second week of FSI through the
final week.  They involve
guest speakers from various industries where data visualization is
used.  

\paragraph{Lunch with Guest Speaker.} Friday, 12:00pm--1:30pm,
Prospect House.  \textit{The lunch with the last speaker will be held
  at Mediterra, a downtown Princeton restaurant.} Students will sign
up to have lunch with one of the five guest speakers at the beginning
of the course. During the selected week, students and the course team
will meet with the guest speaker during a casual, catered lunch.

\section*{Course Requirements}

\begin{itemize}
\item {\bf Class participation (15\%):} Students should actively
  participate in all aspects of the course.  Class participation will
  be judged based on questions asked/answered during the lectures, the
  precepts, and on the online discussion board.  Each portion is
  equally weighted.

\item {\bf Review Questions (15\%):} During the QuantLab, students
  will work on the assigned portion of the textbook and electronically
  submit a small set of {\it Review Questions}.  The answers to {\it
    Review Questions} will be graded pass fail.  Details on these
  assignments are announced at the QuantLab. \textit{This is an
    individual assessment with limited collaboration.}

\item {\bf Problem sets (50\%):} Each week will end with the posting
  of a problem set.  These assignments will be posted on Thursday at
  the end of QuantLab via Blackboard.  Hard copies of your problem
  sets must be turned in at the beginning of the Tuesday
  precept. Electronic submission of your computer code via Blackboard
  must also be done by then. Each problem set will be equally
  weighted. \textit{This is an individual assessment with no
    collaboration.}

\item {\bf Final Project (20\%)}: This is a group data analysis
  project. Students will be assigned to groups. Analyzing a data set
  of their choice, students will write a report of no more than 1,000
  words summarizing a compelling relationship or story they identified
  in the data.  No more than 3 figures/tables can be used.  Details
  regarding the final project will be announced later in the
  course. \textit{This is a group assessment with collaboration
    allowed only within the assigned groups.}

\end{itemize}

\section*{Collaboration Policy}

The assignments in this course are designated as individual or group
assessments. The degree of permissible collaboration depends on the kind of
assignment:

\begin{itemize}
\item {\bf Review Questions.} Students are encouraged to interact with
  each other, the instruction team, and QuantLab tutors in discussing
  their approaches and solutions. This includes conceptual discussion
  and actual computer code. \textit{However, for all other
    assignments, this degree of collaboration is not appropriate!}

\item {\bf Problem Sets.}  No collaboration is allowed. Students may
  ask clarifying questions regarding problem set and midterm questions
  to the instruction team through Piazza. This allows all students to
  benefit from clarifications equally. Clarifying questions about the
  problem sets may not be asked of QuantLab tutors, however.

\item {\bf Final Project.} Students may fully collaborate within their
  assigned groups, and may discuss their group's work with other
  students, the instruction team, and QuantLab tutors.
\end{itemize}

\section*{Plagiarism Policy}

Violations of the above collaboration policy will be treated as instances of
plagiarism. This course will follow a modified version of the guidelines used
for computer science classes here at Princeton.  {\it Please take this guideline
  seriously}.  In the past, plagiarism cases typically result in one-year
suspension from Princeton.

Programming necessitates that you reach your own understanding of the
problem and discover a path to its solution.  {\sc Do not, under any
  circumstances, copy another person's code}. Incorporating someone
else's code into your program in any form is a violation of academic
regulations. Abetting plagiarism or unauthorized collaboration by
sharing your code is also prohibited. Sharing code in digital form is
an especially egregious violation: do not e-mail your code to anyone.

Novices often have the misconception that copying and mechanically
transforming a program (by rearranging independent code, renaming
variables, or similar operations) makes it something different.
Actually, identifying plagiarized source code is easier than you might
think.  For example, there exists computer software that can detect
plagiarism.

This policy supplements the University's academic regulations, making
explicit what constitutes a violation for this course. Princeton
Rights, Rules, Responsibilities handbook asserts:

\begin{quote}
  The only adequate defense for a student accused of an academic
  violation is that the work in question does not, in fact, constitute
  a violation. Neither the defense that the student was ignorant of
  the regulations concerning academic violations nor the defense that
  the student was under pressure at the time the violation was
  committed is considered an adequate defense.
\end{quote}
If you have any questions about these matters, please consult a member of the
instruction team.


\section*{Textbook}

This course uses a draft manuscript of the following textbook.
\begin{quote}
  Imai, Kosuke. {\it Quantitative Social Science: An Introduction}.
  Under contract with Princeton University Press.
\end{quote}
The textbook is made freely available to the students at Blackboard.
Due to the copyright issues, this file should not be distributed to
those who are not taking this class.  

\section*{Statistical Software}

In this course, we use the open-source statistical software \R{}
(\url{http://www.r-project.org}).  \R{} can be more powerful than
other statistical software such as SPSS, STATA and SAS, but it can
also be more difficult to learn.  A variety of resources will be made
available for POL 245 students in order to learn \R{} as efficiently
as possible.  To help make using \R{} easier, we'll be using \Rst{}
(\url{http://www.rstudio.com/})---a user-interface that simplifies
many common operations.

\section*{Get Help} 

Many students will find the materials of this course to be
challenging.  As such, students must seek immediate help when
struggling with the course.  There are several ways, in which students
can get in-person and online help.

\subsection*{In-person Help}

\begin{itemize}
\item Office Hours: 2:00pm to 3:30pm on Fridays in Corwin 127. You
  will be able to ask the instruction team any questions you might
  have about the course materials.  You may also e-mail to set up an
  appointment outside of the office hours.

\item Problem Set Help Sessions: 7:00pm to 9:00pm on Sundays in
  Hargadon G001, G002, and G004 (located in Baker Hall, Whitman
  College) and QuantLab, 3:00pm to 4:30pm on Mondays.  Tutors will not
  give you direct guidance on the actual problem set questions but
  will help you understand the concepts required for solving them.

\item \R{} Drop-in Office Hours: 3:30pm to 4:30pm on Fridays, Butler 
  028.  Dima Gorenshteyn will be available to answer any questions
  about \R{} programming. 

\item \R{} Workshop: 4:30pm to 6:00pm on Fridays, Butler 
  028.  Dima Gorenshteyn will run a short workshop that cover tricky 
  \R{} programming concepts introduced in each week.  The details of a
  workshop will be announced each week. 

\end{itemize}

\subsection*{On-line Help}

In addition to office hours and individual appointments, we will be
available online to answer any questions you may have about the course
materials and the problem sets.  We use the Piazza discussion forum
that will be linked on Blackboard course page or accessible directly
at \url{http://piazza.com}.

Before posting your question, please review previous posts to make
sure that a similar question has not been answered.  {\it In
  accordance with the collaboration policy described above, you should
  not directly post your code for a problem set.}  You should frame
your questions in general terms rather than trying to have us debug
your code directly.  You may subscribe to the Discussion Forum so that
you receive your fellow students' questions and answers to those
questions.  You should also feel free to respond to questions that you
can answer.  Piazza also has a free smartphone application if you are
interested.


\section*{Course Outline}

\subsection*{Introduction: July 21 -- July 22}

During the first two days of the course, you will be introduced to
\R{} statistical programming environment through the use of \Rst.

\textit{Note, the first session on Tuesday, July 21 will be a lecture
  in Sherrerd Hall 101 and not a precept. The second session on
  Wednesday, July 22 will be a precept in the Frist Campus Center. The
  third session on Thursday, July 23 will be a lecture in Sherrerd
  Hall. And, for Friday, July 24 we will hold a precept in Frist
  Campus Center at 10:30 am. These changes affect only the first week of the
  course.}

\begin{center}
  \begin{tabular}{lll}
    \hline
    Lecture & July 21 & Overview of the course \\
    QuantLab & July 21 & Work on Chapter~1; Submit Chapter~1 Review Questions~1~and~2 \\
    Precept & July 22 & Application: Understanding World Population Dynamics\\
    \hline
  \end{tabular}
\end{center}

\bigskip 
\subsection*{Causality:  July 22 -- July 28}

We will learn how to infer causality from data.  We learn the
distinction between randomized experiments and observational studies.
Our applications include the evaluation of strategies for increasing
voter turnout and the effect of class size on educational achievement.

\begin{center}
  \begin{tabular}{lll}
    \hline
    QuantLab~1 & July 22 & Work on Chapter~2 (2.1--2.4); Submit Chapter~2 Review Questions~1 \\
    Lecture~1  & July 23 & Experiments \\
    QuantLab~2 & July 23 & Work on Chapter~2 (2.5--2.7); Submit Chapter~2 Review Questions~2 \\
    Precept~1  & July 24 & Application: Efficacy of Small-class Size in Primary Education\\
    Lecture~2  & July 27 & Observational Studies \\
%    QuantLab~4 & July 27 & Application: Effect of Demographic Change on Exclusionary Attitudes \\
    Precept~2  & July 28 & Application: Success of Leader Assassination as a Natural Experiment \\
    \hline
  \end{tabular}
\end{center}

\bigskip 
\subsection*{Measurement: July 28 -- August 4}

We consider how to measure public opinion using sample surveys.  We
also learn about a measurement strategy regarding latent concepts like
ideology.  Our applications include surveys in Afghanistan and
political polarization in US Congress.

\begin{center}
  \begin{tabular}{lll}
    \hline
    QuantLab~1 & July 28 & Work on Chapter~3 (3.1--3.4); Submit Chapter~3 Review Questions~1 \\
    Lecture~1  & July 29 & Survey Sampling \\
    Precept~1  & July 29 & Application: Political Efficacy in China and Mexico\\
    QuantLab~2 & July 30 & Work on Chapter~3 (3.5--3.8); Submit Chapter~3 Review Questions~2 \\
    Lecture~2  & August 3 & Measurement and Clustering \\
%    QuantLab~7 & August 3 & Application: Changing Minds on Gay Marriage: Revisited \\
    Precept~2  & August 4 & Application: Voting in the United Nations General Assembly
    \\
    \hline
  \end{tabular}
\end{center}

\bigskip 
\subsection*{Prediction: August 4 -- August 13}

We learn about prediction starting with the application of US
presidential election forecasting.  Students will be introduced to
linear regression and how it is related to causality.

\begin{center}
  \begin{tabular}{lll}
    \hline
    QuantLab~1 & August 4 & Work on Chapter~4 (4.1); Submit Chapter~4 Review Questions~1 \\
    Lecture~1  & August 5 & Prediction and Loop \\
    Precept~1  & August 6 & Application: Prediction Based on Betting Markets\\
    QuantLab~2 & August 7 & Work on Chapter~4 (4.2); Submit Chapter~4 Review Questions~2 \\
    Lecture~2  & August 10 & Regression \\
%    QuantLab~10 & August 10 & Application: Ideology of US Supreme Court Justices \\
    Precept~2  & August 11 & Application: Election and Conditional Cash Transfer Program in Mexico
    \\
    QuantLab~11 & August 11 & Work on Chapter~4 (4.3--4.4); Submit Review Questions~3 \\
    Lecture~8  & August 12 & Regression and Causation \\
    Precept~8  & August 13 & Application: Government Transfer and Poverty Reduction in Brazil \\		
    \hline
  \end{tabular}
\end{center}

\bigskip 
\subsection*{Discovery: August 13 -- August 25}

We cover how to analyze three different types of data; textual data,
network data, and spatial data.  Our applications include the
prediction of disputed authorship of The Federalist Papers, the
marriage network in Renaissance Florence, and the expansion of
Wal-mart.

\begin{center}
  \begin{tabular}{lll}
    \hline
    QuantLab~1 & August 13 & Work on Chapter~5 (5.1); Submit Chapter~5 Review Questions~1 \\
    Lecture~1   & August 17 & Textual Data \\
%    QuantLab~13 & August 17 & Application: Predicting Partisanship from Press Releases \\
    Precept~1   & August 18 & Application: Predicting Blog Tone\\
    QuantLab~2 & August 18 & Work on Chapter~5 (5.2); Submit Review Questions~2 \\
    Lecture~2  & August 19 & Network Data \\
    Precept~2  & August 20 & Application: International Trade Network \\
    QuantLab~3 & August 20 & Work on Chapter~5 (5.3); Submit Review Questions~3 \\
    Lecture~3  & August 24 & Spatial Data \\
%    QuantLab~16 & August 24 & Application: Supreme Court Citation Network \\
    Precept~3  & August 25 & Application: Spatial Mapping US Election Results Over Time
    \\
    \hline
  \end{tabular}
\end{center}

\bigskip 
\subsection*{Final Project: August 25 -- August 30}

Students will work on the final project of the course in a small group
with the help from the instruction team and QuantLab tutors.

\begin{center}
  \begin{tabular}{lll}
    \hline
    QuantLab & August 25 & Final Project Work Session \\
    Lecture  & August 26 & Wrapping Up \\
    Precept  & August 27 & Final Project Work Session \\
    QuantLab & August 27 & Final Project Work Session \\
%    QuantLab~19 & August 30 & Final Project Work Session \\
		\hline
	\end{tabular}
\end{center}

\newpage

\renewcommand{\dayname}[1]{
  \ifcase#1 ?
  \or Sun
  \or Mon
  \or Tue
  \or Wed
  \or Thur
  \or Fri
  \or Sat
  \else ?
  \fi
}


\section*{POL 245 Schedule}

Due to the condensed nature of the course, the semester and weekly
schedules (including assignments) can seem very full. To help with
this, we have a Google Calendar with events, due dates, and sessions.
The URL for this calendar is \url{http://goo.gl/D3pKXL}.  We also
outline a typical week of assignments in POL 245 below:

\vspace{.5in}\hspace{-0.5in}\noindent\begin{minipage}{7.5in}
  
  \StartingDayNumber=1
  
  \begin{calendar}{\linewidth}
      
    \day{}{
      \textbf{7:00pm-9:00pm} \daysep Problem Set Help Session \\[3pt]
    }
    
    \day{}{
      \textbf{1:30pm-2:30pm} \daysep Lecture \\[3pt]
      \vfill       
      \textbf{3:00pm-4:30pm} \daysep Problem Set Help Session \\[3pt]
    }
    
    \day{}{
      \textbf{1:30pm} \daysep Problem Set deadline \\[3pt]
      \vfill 
      \textbf{1:30pm-2:50pm} \daysep Precept \\[3pt]
      \vfill
      \textbf{3:00pm-4:30pm} \daysep QuantLab \\[3pt]
    }
    
    \day{}{
      \textbf{1:30pm-2:30pm} \daysep Lecture \\[3pt] 
    }
    
    \day{}{
      \textbf{1:30pm-2:50pm} \daysep Precept \\[3pt]
      \vfill    
      \textbf{3:00pm-4:30pm} \daysep QuantLab \\[3pt]
      \textbf{4:30pm} \daysep New Problem Set posted \\[3pt]
      \vfill
    } 

    \day{}{
      \textbf{10:30am-11:50am} \daysep Guest Speaker \\[3pt]
      \vfill 
      \textbf{2:00pm-3:30pm} \daysep Office hours \\[3pt]       
      \vfill 
      \textbf{3:30pm-4:30pm} \daysep \R{} Office hours \\[3pt]       
      \vfill 
      \textbf{4:30pm-6:00pm} \daysep \R{} Workshop \\[3pt]       
    }

    \day{}{}
  \end{calendar}
  
  % \vfill
  \end{minipage}

\section*{Guest Speakers}

For five of the six weeks during the course, we will have guest
speakers during our Friday sessions. In addition to their presentation
from 10:30am -- 11:50am in Wallace Hall 300, they will be on campus
for lunch with a small group of students. Students will sign up for
lunch with a specific speaker at the beginning of the course.

\small
\begin{center}
  \renewcommand{\arraystretch}{1.2}
  \begin{tabular}{c c c c}
    \hline
    \bf FSI  Week & \bf Date & \bf Guest                                     & \bf Affiliation    \\
    \hline
    2             & 7/31     & \parbox{3in}{\vspace{.1cm} Rebecca Lai  \\ \it Graphics Intern              \\      
      \sc Visualizing the News}                                         & New York Times     \\
    ~ \\
    3             & 8/7      & \parbox{3in}{Aaron Strauss \\ \it Executive Director                    \\
      \sc How Campaigns Use Analytics and Experiments to Influence Voters}                        & Analyst Institute           \\ 
    ~ \\
    4             & 8/14      & \parbox{3in}{Dan Chapsky \\ \it Advertisement Data Scientist \\
      \sc Truth, Beauty and Social Data: Using Open Data Online to Predict Offline Events} & Facebook \\
    ~ \\
    5             & 8/21     & \parbox{3in}{Elizabeth Roodhouse \\ \it Social Scientist \\
      \sc Big Data at Google \& YouTube:  Exploring Trends in the Online Video Landscape}                                       & Google \\
    ~ \\
    6             & 8/28     & \parbox{3in}{Neil Paine \\ \it Senior Sportswriter \\
    	\sc Sports and Data Journalism in the Post-Moneyball Era \vspace{.1cm}}                                       & FiveThirtyEight \\
    \hline
  \end{tabular}
\end{center}


\clearpage


\end{document}



%%%%%%%%%%%%%%%%%%%%%%%%%%%%%%%%%%%%%%%%%%%%%%%%%%%%%%%%%%%%%%%%%%%%%%%%%%%%%%% 
%%%%%%%%%%%%%%%%%%%%%%%%%%%%%%%%%%%%%%%%%%%%%%%%%%%%%%%%%%%%%%%%%%%%%%%%%%%%%%% 
%%%%%%%%%%%%%%%%%%%%%%%%%%%%%%%%%%%%%%%%%%%%%%%%%%%%%%%%%%%%%%%%%%%%%%%%%%%%%%% 

