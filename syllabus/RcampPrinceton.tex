\documentclass[12pt]{article}
\usepackage[left=1in,top=.8in,right=1in,bottom=.8in,nohead]{geometry}
\usepackage{graphicx}
\usepackage{hyperref}
\begin{document}
\thispagestyle{empty}
\begin{center}
  {\bf \Large{{Statistical Programming Camp \\ Spring 2017}}}\\
  \vspace{.1in}
   Sunday, January 29 -- Friday,  February 3 \\
    Morning Session: 10:00
    AM -- 12:00 PM\\
   Afternoon Session: 1:00 -- 3:00 PM \\ Wallace 300 (Sun-Thurs); Sherrerd 101 (Friday)
\end{center}

%\setlength{\unitlength}{1in}

%% Software Package Names
\newcommand{\R}{\textsf{\textbf{R}}}
\newcommand{\QSS}{\textsf{QSS}}
\newcommand{\Rst}{\textsf{\textbf{RStudio}}}


\hrulefill

\begin{tabular}{ll}
{\bf Instructors:}  & Munji Choi \& Asya Magazinnik \\
Office:& Corwin 026 (Munji), Robertson 323 (Asya) \\
Email:  & Munji: munjic@princeton.edu\\
& Asya: asyam@princeton.edu\\
Office Hours:&  3:30 -- 4:30 PM (Corwin 023) \\
{\bf Faculty advisor:}& Kosuke Imai
\end{tabular}


\hrulefill

\paragraph{Description:} This camp will prepare students for POL 572
and other quantitative analysis courses offered in the Politics
department and elsewhere.  Although participation in this camp is
completely voluntary, the materials covered in this camp are a
pre-requisite for POL 572.  Students will learn the basics of
statistical programming using \R, an open-source computing
environment.  Using data from published journal articles, students
will learn how to manipulate data, create graphs and tables, and
conduct basic statistical analysis. This camp assumes knowledge of
probability and statistics as covered in POL 571.

\paragraph{Structure:} The camp will meet for five and a half days
starting with an afternoon session on Sunday and continuing with two
daily sessions (morning and afternoon) each following day. Each
session lasts two hours with an hour break between. The first hour of
the morning session is lecture-based; the other half will be devoted
to instructors and students working on programming exercises
together. In the afternoon session, students will assemble in their
assigned groups to solve a daily problem set, and instructors will be
present to help with questions. Students who finish early are free to
leave the day's afternoon session. After the lecture, students are
required to go over the assigned readings for the following day. We
may devote more or less time to lecture depending on the day's
materials.

We will start with the afternoon Sunday session and will finish on
Friday afternoon.  The content is organized into 6 modules, each
containing sessions that cover material students will need for the
module's problem set. In the afternoon session, we will also take a
short amount of time to review the graded problem set from the
previous module.

\paragraph{Discussion Board:} 
In addition to precepts and office hours, please use the
\textbf{\textsf Piazza} Discussion Board at \url{https://piazza.com/}
when asking questions about lectures, problem sets, and other course
materials.  This allows all students to benefit from the discussion
and to help each other understand the materials. Both students and
instructors are encouraged to participate in discussions and answer
any questions that are posted.

To join the Piazza site, click on ``Search Your Classes" from
the Piazza homepage.  After specifying Princeton University as your
school, search for ``Statistical Programming Camp.'' You will
then be prompted to enter your {\tt princeton.edu} email address to
confirm your registration. Piazza can also be accessed from within
Blackboard by going to the Programming Camp course page and clicking on the
link to ``Piazza Messageboard.''  In addition, all class announcements
will be made through Piazza. Blackboard will still be used for hosting
all class materials. 

Some useful tips for Piazza include: 
\begin{itemize}
\item Piazza  has apps available for the iOS and Android platforms. The apps are free downloads and provide complete access to all of Piazza's message board features.

\item To insert \LaTeX-formatted text in a post, place a double dollar sign (\$\$) on both ends of the relevant text, or click the $fx$ button in the Details toolbar above your post.

\item To add formatted \textbf{\textsf{R}} code to a post, click the
  ``pre" button in the Details toolbar above your post. A grey text
  box will open up where you can paste code from \textbf{\textsf{R}}.

\item You can classify a post using pre-selected tags, or you can generate your own by prepending a hash (\#) to 
your chosen label. Posts can then be sorted by these tags using the search bar in the left-hand column.

\end{itemize}

\paragraph{Materials and Website:} Students are encouraged to bring
their personal laptop to each session. The required textbook for the
course is:
\begin{center}
  Imai, Kosuke. {\it Quantitative Social Science: An Introduction}
  (\QSS).
\end{center}
forthcoming from the Princeton University Press in 2017. The textbook
is made freely available to students via Blackboard. Due to
copyright issues, this file should not be distributed to those who are not taking this class.

\paragraph{Assignments:} The only way to learn statistics is by doing.
To ensure steady and efficient learning, we assign daily problem sets
and a final exam. The final exam and problems sets will be assessed
and will count towards a final grade with the following weights:

\begin{flushleft}
\begin{tabular}{ll}
Problem Sets: & 70 \% (4 equally weighted assignments, completed with assigned group)\\
Final Exam: & 30 \% (Individual, no collaboration) \\
\end{tabular}
\end{flushleft}

\noindent We ask you to submit your solutions to the problem sets in
the appropriate folders at Statistical Programming Camp Blackboard by 12 AM.
The final exam will be due electronically by 10 pm on
Saturday, February $4^{th}$. \\

\noindent Students must complete the required reading before each day's session. For the first module, students must have read Chapter 1 in the required textbook, completed exercise 1.4 at the end of the chapter, and registered for Piazza. The first chapter provides students with a guide to install \R, and helps them get familiarized with the programming language.

\paragraph{Group-based Learning:} To promote learning and collaboration, students are assigned to a 
group of three students. Groups are required to work together on
in-class problem  sets, and all group members should contribute
equally to all assignments. Students hand in problem sets, as a group, with the name of each group member on the assignment.\\

\noindent Groups have two options when completing problem sets. Group members can work together, from start to finish, or work individually on the entire problem set, then meet as a group to discuss the solutions and write the final submission. Dividing the work among group members is not allowed, since the goal of problem sets is to ensure every individual learns the material. The final grade is based on both the group problem 
set score and your individual performance on the final exam. There is to be no collaboration between groups, 
aside from public posts on Piazza.\\

\noindent All exercises are available at the end of each of the module's required chapter. Additional class materials, including solutions to graded exercises, will be made available through the Statistical Programming Camp Blackboard 
site under the Course Materials link.% Edit this sentence (Should we refer student's to https://github.com/kosukeimai/qss for problem sets and exercises?

\paragraph{Camp Outline:}

%\begin{center} \begin{minipage}{6.5in}
\begin{description}
\item[Module 1] (Sunday Afternoon)\hfill \\
Topic:  Introduction to \R\\% Or something like Intro to object-oriented programming
Required Reading: Chapter 1 \\
%Exercise 1.4.1 due 1 PM, Sunday \\ 

\item[Module 2] (Monday Morning/Afternoon)\hfill \\
Topic: Causality\\
Required Reading: Chapter 2 \\
%Exercise 2.8.3 due 12 AM, Tuesday\\

\item[Module 3] (Tuesday Morning/Afternoon) \hfill \\
Topic: Measurement\\
Required Reading: Chapter 3\\
%Exercise 3.9.3  due 12 AM, Wednesday\\

\item[Module 4] (Wednesday Morning/Afternoon)\hfill \\
Topic: Prediction \\
Required Reading: Chapter 4 \\
%Exercise 4.5.1 due 12 AM, Thursday\\

\item[Module 5] (Thursday Morning/Afternoon) \hfill \\
Topic: Probability\\
Required Reading: Chapter 6\\
%Exercise 6.7.3 due 12 AM, Friday\\

\item[Module 6] (Friday Morning/Afternoon) \hfill \\
Topic: Uncertainty\\
Required Reading: Chapter 7\\
Lunch Party: 12.00--1.00 pm\\
%Exercise 7.9.1 due 10 PM, Saturday\\
\end{description}
%\end{minipage}
%\end{center}


\end{document} 


\end{description}